% Template for PLoS
% Version 3.4 January 2017
%
% % % % % % % % % % % % % % % % % % % % % %
%
% -- IMPORTANT NOTE
%
% This template contains comments intended 
% to minimize problems and delays during our production 
% process. Please follow the template instructions
% whenever possible.
%
% % % % % % % % % % % % % % % % % % % % % % % 
%
% Once your paper is accepted for publication, 
% PLEASE REMOVE ALL TRACKED CHANGES in this file 
% and leave only the final text of your manuscript. 
% PLOS recommends the use of latexdiff to track changes during review, as this will help to maintain a clean tex file.
% Visit https://www.ctan.org/pkg/latexdiff?lang=en for info or contact us at latex@plos.org.
%
%
% There are no restrictions on package use within the LaTeX files except that 
% no packages listed in the template may be deleted.
%
% Please do not include colors or graphics in the text.
%
% The manuscript LaTeX source should be contained within a single file (do not use \input, \externaldocument, or similar commands).
%
% % % % % % % % % % % % % % % % % % % % % % %
%
% -- FIGURES AND TABLES
%
% Please include tables/figure captions directly after the paragraph where they are first cited in the text.
%
% DO NOT INCLUDE GRAPHICS IN YOUR MANUSCRIPT
% - Figures should be uploaded separately from your manuscript file. 
% - Figures generated using LaTeX should be extracted and removed from the PDF before submission. 
% - Figures containing multiple panels/subfigures must be combined into one image file before submission.
% For figure citations, please use "Fig" instead of "Figure".
% See http://journals.plos.org/plosone/s/figures for PLOS figure guidelines.
%
% Tables should be cell-based and may not contain:
% - spacing/line breaks within cells to alter layout or alignment
% - do not nest tabular environments (no tabular environments within tabular environments)
% - no graphics or colored text (cell background color/shading OK)
% See http://journals.plos.org/plosone/s/tables for table guidelines.
%
% For tables that exceed the width of the text column, use the adjustwidth environment as illustrated in the example table in text below.
%
% % % % % % % % % % % % % % % % % % % % % % % %
%
% -- EQUATIONS, MATH SYMBOLS, SUBSCRIPTS, AND SUPERSCRIPTS
%
% IMPORTANT
% Below are a few tips to help format your equations and other special characters according to our specifications. For more tips to help reduce the possibility of formatting errors during conversion, please see our LaTeX guidelines at http://journals.plos.org/plosone/s/latex
%
% For inline equations, please be sure to include all portions of an equation in the math environment.  For example, x$^2$ is incorrect; this should be formatted as $x^2$ (or $\mathrm{x}^2$ if the romanized font is desired).
%
% Do not include text that is not math in the math environment. For example, CO2 should be written as CO\textsubscript{2} instead of CO$_2$.
%
% Please add line breaks to long display equations when possible in order to fit size of the column. 
%
% For inline equations, please do not include punctuation (commas, etc) within the math environment unless this is part of the equation.
%
% When adding superscript or subscripts outside of brackets/braces, please group using {}.  For example, change "[U(D,E,\gamma)]^2" to "{[U(D,E,\gamma)]}^2". 
%
% Do not use \cal for caligraphic font.  Instead, use \mathcal{}
%
% % % % % % % % % % % % % % % % % % % % % % % % 
%
% Please contact latex@plos.org with any questions.
%
% % % % % % % % % % % % % % % % % % % % % % % %

\documentclass[10pt,letterpaper]{article}
\usepackage[top=0.85in,left=2.75in,footskip=0.75in]{geometry}

% amsmath and amssymb packages, useful for mathematical formulas and symbols
\usepackage{amsmath,amssymb}

% Use adjustwidth environment to exceed column width (see example table in text)
\usepackage{changepage}

% Use Unicode characters when possible
\usepackage[utf8x]{inputenc}

% textcomp package and marvosym package for additional characters
\usepackage{textcomp,marvosym}

% cite package, to clean up citations in the main text. Do not remove.
\usepackage{cite}

% Use nameref to cite supporting information files (see Supporting Information section for more info)
\usepackage{nameref,hyperref}

% line numbers
\usepackage[right]{lineno}

% ligatures disabled
\usepackage{microtype}
\DisableLigatures[f]{encoding = *, family = * }

% color can be used to apply background shading to table cells only
\usepackage[table]{xcolor}

% array package and thick rules for tables
\usepackage{array}

% create "+" rule type for thick vertical lines
\newcolumntype{+}{!{\vrule width 2pt}}

% create \thickcline for thick horizontal lines of variable length
\newlength\savedwidth
\newcommand\thickcline[1]{%
  \noalign{\global\savedwidth\arrayrulewidth\global\arrayrulewidth 2pt}%
  \cline{#1}%
  \noalign{\vskip\arrayrulewidth}%
  \noalign{\global\arrayrulewidth\savedwidth}%
}

% \thickhline command for thick horizontal lines that span the table
\newcommand\thickhline{\noalign{\global\savedwidth\arrayrulewidth\global\arrayrulewidth 2pt}%
\hline
\noalign{\global\arrayrulewidth\savedwidth}}


% Remove comment for double spacing
%\usepackage{setspace} 
%\doublespacing

% Text layout
\raggedright
\setlength{\parindent}{0.5cm}
\textwidth 5.25in 
\textheight 8.75in

% Bold the 'Figure #' in the caption and separate it from the title/caption with a period
% Captions will be left justified
\usepackage[aboveskip=1pt,labelfont=bf,labelsep=period,justification=raggedright,singlelinecheck=off]{caption}
\renewcommand{\figurename}{Fig}

% Use the PLoS provided BiBTeX style
\bibliographystyle{plos2015}

% Remove brackets from numbering in List of References
\makeatletter
\renewcommand{\@biblabel}[1]{\quad#1.}
\makeatother

% Leave date blank
\date{}

% Header and Footer with logo
\usepackage{lastpage,fancyhdr,graphicx}
\usepackage{epstopdf}
\pagestyle{myheadings}
\pagestyle{fancy}
\fancyhf{}
\setlength{\headheight}{27.023pt}
\lhead{\includegraphics[width=2.0in]{PLOS-submission.eps}}
\rfoot{\thepage/\pageref{LastPage}}
\renewcommand{\footrule}{\hrule height 2pt \vspace{2mm}}
\fancyheadoffset[L]{2.25in}
\fancyfootoffset[L]{2.25in}
\lfoot{\sf PLOS}

%% Include all macros below

\newcommand{\lorem}{{\bf LOREM}}
\newcommand{\ipsum}{{\bf IPSUM}}

%% END MACROS SECTION


\begin{document}
\vspace*{0.2in}

% Title must be 250 characters or less.
\begin{flushleft}
{\Large
\textbf\newline{Hub connectivity and gene expression in the worm connectome} % Please use "sentence case" for title and headings (capitalize only the first word in a title (or heading), the first word in a subtitle (or subheading), and any proper nouns).
}
\newline
% Insert author names, affiliations and corresponding author email (do not include titles, positions, or degrees).
\\
Name1 Surname\textsuperscript{1,2\Yinyang},
Name2 Surname\textsuperscript{2\Yinyang},
Name3 Surname\textsuperscript{2,3\textcurrency},
Name4 Surname\textsuperscript{2},
Name5 Surname\textsuperscript{2\ddag},
Name6 Surname\textsuperscript{2\ddag},
Name7 Surname\textsuperscript{1,2,3*},
with the Lorem Ipsum Consortium\textsuperscript{\textpilcrow}
\\
\bigskip
\textbf{1} Affiliation Dept/Program/Center, Institution Name, City, State, Country
\\
\textbf{2} Affiliation Dept/Program/Center, Institution Name, City, State, Country
\\
\textbf{3} Affiliation Dept/Program/Center, Institution Name, City, State, Country
\\
\bigskip

% Insert additional author notes using the symbols described below. Insert symbol callouts after author names as necessary.
% 
% Remove or comment out the author notes below if they aren't used.
%
% Primary Equal Contribution Note
\Yinyang These authors contributed equally to this work.

% Additional Equal Contribution Note
% Also use this double-dagger symbol for special authorship notes, such as senior authorship.
\ddag These authors also contributed equally to this work.

% Current address notes
\textcurrency Current Address: Dept/Program/Center, Institution Name, City, State, Country % change symbol to "\textcurrency a" if more than one current address note
% \textcurrency b Insert second current address 
% \textcurrency c Insert third current address

% Deceased author note
\dag Deceased

% Group/Consortium Author Note
\textpilcrow Membership list can be found in the Acknowledgments section.

% Use the asterisk to denote corresponding authorship and provide email address in note below.
* correspondingauthor@institute.edu

\end{flushleft}
% Please keep the abstract below 300 words
\section*{Abstract}
Text


% Please keep the Author Summary between 150 and 200 words
% Use first person. PLOS ONE authors please skip this step. 
% Author Summary not valid for PLOS ONE submissions.   
\section*{Author summary}
Text

\linenumbers

% Use "Eq" instead of "Equation" for equation citations.
\section*{Introduction}

\begin{enumerate}
    \item{Structural network properties}
    \begin{itemize}
    \item{rich-club, hubs, conservation cross-scale/species/methods}
    \item{Prior work in worm: different types of connectomes - wired/unwired, hubs}
    \end{itemize}    

    \item{Structure-expression}
    \begin{itemize}
    \item{Datasets in different scales: mouse/rat, human (also heritability), developmental, large-scale datasets}
    \item{Kaufman example \cite{Kaufman2006}, Baruch 2008 \cite{Baruch2008b}, Varadan 2006 \cite{Varadan2006} also relevant.}
    \end{itemize}  
    
    \item{Hubs plus gene expression}
    \begin{itemize}
    \item{groundbreaking (and also breathtaking) work by Ben and Alex \cite{Fulcher2015}}
    \item{Different types of connections - different expression}
    \end{itemize}  
    \item{Summary}
    
 \end{enumerate}

Connectivity in brain networks is not uniformly distributed. Network elements with high node degree – i.e., a large number of connections to other areas – are called `hubs'.
When hubs are more densely interconnected than expected by chance they form a `rich-club', the idea being that the richest members of the network (in terms of connections) are tightly connected to each other, thus forming a club.
These densely interconnected hubs are thought to promote efficient integration between anatomically distinct areas and play an important role in brain functioning.
It has been shown that hubs exhibit distinct transcriptional signatures in both humans \cite{Vertes2016a} and mice \cite{Fulcher2015}.
According to Fulcher \& Fornito \cite{Fulcher2015}, connections involving rich club hubs carry a distinctive genetic signature, which is driven by genes regulating the synthesis and breakdown of adenosine triphosphate (ATP) – the primary energetic substrate of neuronal signaling \cite{Fulcher2015}. These findings highlight a close relationship between metabolic expenditure and the high signaling load of hub regions in the brain, as has been previously proposed \cite{Bullmore2012}.
We therefore have some preliminary indications that the transcriptional signature of hubs may be a consistent feature of mammalian brain networks, but it is not known how distinctive this expression signature is; and in particular, whether it holds true for networks resolved at the scale of individual neurons and synapses.
To test this possibility, we aimed to replicate findings presented in \cite{Fulcher2015} using microscale connectivity data in C. elegans and gene expression data from WormBase.
We sought to determine whether hubs in the C. elegans connectome exhibit distinct gene expression patterns.\\

% \begin{eqnarray}
% \label{eq:schemeP}
%     \mathrm{P_Y} = \underbrace{H(Y_n) - H(Y_n|\mathbf{V}^{Y}_{n})}_{S_Y} + \underbrace{H(Y_n|\mathbf{V}^{Y}_{n})- H(Y_n|\mathbf{V}^{X,Y}_{n})}_{T_{X\rightarrow Y}},
% \end{eqnarray}


% METHODS SECTION
\section*{Materials and methods}

% For figure citations, please use "Fig" instead of "Figure".
\begin{itemize}
    \item{Connectivity data: types of synapses, weighted/binary, degree; Rich club analysis, how neurons are annotated}
    \item{Expression data: where data comes from, processing (options: annotations, qualifiers)}
    \item{Coexpression metric: how did we choose? How different measures depend on the number of genes; The effect of coexpression/space: can not be corrected for; Excluding left/right homolog gene expression from calculations}
    \item{Enrichment: software (ermineJ), how to score genes (maybe define gene scoring within results as scoring for connected and unconnected and rich/feeder vs peripheral is different)}
    
\end{itemize}    
% METHODS: Outline (add what it's about)
To investigate the relationship between hub connectivity and gene expression in a micro-scale network we coupled two publicly available datasets containing synaptic-level connectivity network and gene expression signatures for the somatic nervous system of the \textit{C. elegans} hermaphrodite.\\

% METHODS: connectivity data
\textbf{Connectivity data}. Full neuron level connectivity data for 279 somatic neurons (282 nonpharyngeal neurons excluding VC6 and CANL/R neurons, which are missing connectivity data) with corresponding 2D coordinates was provided by Varshney et al. \cite{Varshney2011}.
To avoid any possible inherent differences based on connectivity type we focused on binary chemical synapse network (M=1961 connections).
In the case of chemical synapses identities of presynaptic and postsynaptic neurons are well-defined, therefore each neuron is characterised with two binary vectors: one for incoming, another for outgoing connections, resulting in a directed connectivity matrix.
Total number of connections involving a given neuron is defined by its degree \textit{k} - the sum of incoming and outgoing connections.

% METHODS: rich club analysis
\textbf{Rich club.}.  
    
    
% METHODS: gene expression
\textbf{Gene expression.}
(http://wormbase.org, release WS256),
with a gene expression dataset containing binary expression profiles for 932 genes (WormBase WS254 release; from ~15 000 genes queried, 932 genes were directly (specifically) annotated to at least one neuron).
Each neuron has between 3 and 135 genes expressed (median=20); each gene is expressed in a number of neurons ranging from 1 to 148 (median$=$4).
It is important to note that, because the absence of expression data not recorded in the database, it is not possible to distinguish between the following two cases: (i) ``gene is not expressed'' and (ii) ``there is no information on whether gene is expressed''.
Both cases are recorded and analyzed as ``0'' expression here.\\

% METHODS: gene coexpression
\textbf{Gene coexpression.}\\

% METHODS: gene enrichment
\textbf{Gene enrichment.}


 \begin{figure}[!h]
 \caption{{\bf Schematic representation of the work-flow}
Connectivity matrix coded according to link type (and probably gene expression matrix: rows as it is (neuron order), columns ordered according to similarity).}
 \label{SchematicRepresentation}
 \end{figure}

To measure how similar gene expression patterns are between pairs of neurons (`gene coexpression' or `transcriptional similarity' across 932 genes), we computed the mean square contingency coefficient, $\phi$, between their binary expression profiles, as shown in Figure 2.
A coexpression value of 1 means that expression profiles for both neurons are exactly matching; a coexpression of 0 means that gene expression between a pair of neurons is matching in half the cases and mismatching in the other half; and a coexpression value of $-1$ means that gene expression values are mismatching in all cases (i.e., if a gene is expressed in one neuron, it will not be expressed in the other one, and vice versa, for all genes).

In the C. elegans connectome, 92 neurons have bilateral representation.
Gene coexpression values for these pairs range between 0.81 to 1, with a mean of 0.98.
To ensure that our results are not influenced by this symmetry, coexpression values between the pairs of those symmetrical neurons were excluded from all further analyses.


% Place figure captions after the first paragraph in which they are cited.
% \begin{figure}[!h]
% \caption{{\bf Bold the figure title.}
% Figure caption text here, please use this space for the figure panel descriptions instead of using subfigure commands. A: Lorem ipsum dolor sit amet. B: Consectetur adipiscing elit.}
% \label{fig1}
% \end{figure}

% Results and Discussion can be combined.
\section*{Results}
\subsection*{Connected vs unconnected}

\begin{itemize}
    \item{difference in coexpression}
    \item{Enrichment for this part: what genes driving the difference}
\end{itemize} 
%% for now figres are old - just to define where the figure is
First, we compared gene coexpression between connected and unconnected neurons in order to determine if connected neurons have more similar gene expression patterns than unconnected neurons.

The distribution of transcriptional similarity, $\phi$, between all pairs of neurons, grouped by connected and unconnected pairs, is plotted in Figure 3.
Connected pairs of neurons have more similar expression profiles than unconnected pairs (Welch's t test, P $ \leq 10^{-21}$ ). To investigate which functional groups of genes contributed to this difference in transcriptional coupling between connected and unconnected neurons, we used a measure that quantifies the contribution of each gene (gene contribution score (GCC)).
This statistic quantifies how likely it is that a given gene shows matching expression in connected pairs of neurons than expected by chance given its expression profile.

 \begin{figure}[!h]
 \caption{{\bf Coexpression between connected and unconnected neurons}
Distribution of transcriptional similarity for all pairs of neurons, grouped by connected and unconnected pairs (Welch’s t test, P $ \leq 10^{-21}$). Note that symmetric pairs of neurons with high coexpression (92 pairs) have been excluded from this analysis.}
 \label{ConUnconChemical}
 \end{figure}

These GCC scores were then used to perform a gene function enrichment analysis for Gene Ontology (GO) biological processes using gene score resampling (GSR) implemented in ermineJ \cite{Gillis2010}.
In this analysis average gene scores for a specific GO category are compared to a randomly selected gene sample of the same size.
Therefore, the resulting $P$-value (last column in Figure 4) quantifies how unlikely it is to find genes with high scores in a specific GO category compared to chance.
The majority of the significant ($P < 0.05$) GO categories are related to neuronal connectivity and communication, as listed in Figure 4 (second column: Description).
These findings are in line with our expectations as they confirm that genes that are more likely to be expressed in a pair of connected neurons are in fact related to neuronal connectivity and communication.

\subsection*{Coexpression: link type}
\begin{itemize}
    \item{Coexpression for rich/feeder/peripheral links}
    \item{Enrichment for links involving hubs (rich/feeder), define how genes were scored here}
\end{itemize} 

For the second part of our analysis, we focused on features of rich-club organization of C. elegans connectome. We first confirmed the rich-club organization of the C. elegans connectome, in which a small number (n=14) of high degree nodes (degree, $k = 41-58$) are more densely inter-connected than expected by chance, consistent with prior analysis \cite{Towlson2013} (see Figure 5).
The highest degree neurons are mostly interneurons, with the exception of one motor neuron (Table 1).
Neuron descriptions from WORMATLAS reveal that 10 of these interneurons are ventral cord interneurons.

We investigated the relationship between gene coexpression and neuronal connectivity by comparing three different classes of neuron pairs.
At each level of node degree, $k$ (x-axis in Figure 5), we labeled each neuron as either a hub (nodes with degree $\geq k$) or a nonhub (otherwise), and then labeled each connection as rich (hub - hub), feeder (nonhub - hub or hub - nonhub), or peripheral (nonhub- nonhub). Across the topological rich club regime, mean gene coexpression is significantly higher for rich connections (red in Figure 6) and increases slightly for feeder links from $k=52$.

 \begin{figure}[!h]
 \caption{{\bf Mean coexpression for rich, feeder and peripheral links as a function of degree.}
Top: degree distribution. Mean gene coexpression for rich, feeder, and peripheral connections as a function of \textit{k}, with the mean across all network links shown as a dashed black line and the topological rich club regime shaded grey. Circles indicate a statistically significant increase in gene coexpression in a given link type relative to the rest of the network (one-sided Welch’s t test; $P < 0.05$}
 \label{RCdegree}
 \end{figure}

To see how coexpression values are distributed for different types of links, we show a separate case in Figure 7, in the topological rich-club regime, where hubs are defined as having at least 42 connections ($k \geq 42$).
Here we can see a gradient of coexpression between rich/feeder/peripheral, similar to that found for the mouse \cite{Fulcher2015}.

\begin{figure}[!h]
 \caption{{\bf Mean coexpression for rich, feeder and peripheral links as a function of degree.}
Coexpression ($\phi$) for different types of network connections, where hubs are neurons with degree $k \geq 42$. P-values are from pairwise comparisons using Welch’s t test.}
 \label{ViolinPlots}
 \end{figure}


To investigate which functional groups of genes contributed to the higher coexpression for links involving hubs (rich and feeder links, see Figure 6) we calculated a variation of a GCC which assigns each gene a score according to the probability of getting a higher number of matches in gene expression for rich and feeder links.
Lower probability in this case means that a gene has a high number of matches for rich and feeder links in the experimental data, and therefore is of high interest for the analysis.
Enrichment analysis show that the majority of the significant ($P < 0.05$) GO categories are related to glutamate related signaling, neuronal connectivity and communication as well as neurotransmitter metabolic processes (second column: Description in Figure 8).
These findings are in line with our expectations as high degree neurons are highly involved in signaling and are likely to use glutamate and acetylcholine.

\subsection*{EXTRAS: neuron type, lineage}
\begin{itemize}
    \item{Difference between interneurons/motor/sensory neurons - effect driven by interneurons}
    \item{Maybe include lineage data: lineage distance increases with degree for feeder links; no difference between rich and peripheral links; Sort of confirms separation into rich/feeder/peripheral links as distance increases for feeders?}
\end{itemize} 

While above we demonstrated a relationship between transcriptional similarity and structural connectivity, we note that hubs of the C. elegans connectome are dominated by interneurons.
Correspondingly, we investigated whether the effect of increasing gene coexpression is specific to interneurons.
The relationship between average coexpression and degree for connections including different types of neurons (interneurons, motor neurons and sensory neurons) is shown in Figure 8.
Connections involving interneurons show a pattern of increasing average coexpression with increasing degree, however neither connections involving motor nor sensory neurons resemble this trend; even showing evidence for the opposite relationship.
In other words, motor and sensory neurons with higher degree show less similar gene expression patterns.

\begin{figure}[!h]
 \caption{{\bf Coexpression for different types of neurons. }
Top: Average gene coexpression as a function of degree for connections involving different types of neurons. Bottom: The number of connections in each category for a range of degrees. }
 \label{NeuronTypePlot}
 \end{figure}
Top: Average gene coexpression as a function of degree for connections involving different types of neurons. Bottom: The number of connections in each category for a range of degrees. 


% Place tables after the first paragraph in which they are cited.
% \begin{table}[!ht]
% \begin{adjustwidth}{-2.25in}{0in} % Comment out/remove adjustwidth environment if table fits in text column.
% \centering
% \caption{
% {\bf Table caption Nulla mi mi, venenatis sed ipsum varius, volutpat euismod diam.}}
% \begin{tabular}{|l+l|l|l|l|l|l|l|}
% \hline
% \multicolumn{4}{|l|}{\bf Heading1} & \multicolumn{4}{|l|}{\bf Heading2}\\ \thickhline
% $cell1 row1$ & cell2 row 1 & cell3 row 1 & cell4 row 1 & cell5 row 1 & cell6 row 1 & cell7 row 1 & cell8 row 1\\ \hline
% $cell1 row2$ & cell2 row 2 & cell3 row 2 & cell4 row 2 & cell5 row 2 & cell6 row 2 & cell7 row 2 & cell8 row 2\\ \hline
% $cell1 row3$ & cell2 row 3 & cell3 row 3 & cell4 row 3 & cell5 row 3 & cell6 row 3 & cell7 row 3 & cell8 row 3\\ \hline
% \end{tabular}
% \begin{flushleft} Table notes Phasellus venenatis, tortor nec vestibulum mattis, massa tortor interdum felis, nec pellentesque metus tortor nec nisl. Ut ornare mauris tellus, vel dapibus arcu suscipit sed.
% \end{flushleft}
% \label{table1}
% \end{adjustwidth}
% \end{table}


%PLOS does not support heading levels beyond the 3rd (no 4th level headings).

% \subsubsection*{3rd level heading}
% Text.
%
% \begin{enumerate}
%     \item{react}
%     \item{diffuse free particles}
%     \item{increment time by dt and go to 1}
% \end{enumerate}
%
%
% \subsection*{Another subsection}
%
%
% \begin{itemize}
%     \item First bulleted item.
%     \item Second bulleted item.
%     \item Third bulleted item.
% \end{itemize}

\section*{Discussion}
\begin{itemize}
    \item{First coexpression analysis in worm. Despite noisy gene expression data we get some insights into the genetic basis of connectivity on a neuronal level}
\end{itemize} 
Discussion text
\section*{Limitations}

\begin{itemize}
    \item{Binary gene expression data}
    \item{No way of discriminating between missing data and the absence of expression}
    \item{only around 5 percent of genes in the genome available}
    \item{annotation problems: different qualifiers, loosing sensitivity/specificity if including too much or too little - need to balance}

\end{itemize} 

\section*{Conclusion}

Conclusion text. 

\section*{Supporting information}

% Include only the SI item label in the paragraph heading. Use the \nameref{label} command to cite SI items in the text.
\paragraph*{S1 Fig.}
\label{S1_Fig}
{\bf Bold the title sentence.} Add descriptive text after the title of the item (optional).

\paragraph*{S2 Fig.}
\label{S2_Fig}
{\bf Lorem ipsum.} Maecenas convallis mauris sit amet sem ultrices gravida. Etiam eget sapien nibh. Sed ac ipsum eget enim egestas ullamcorper nec euismod ligula. Curabitur fringilla pulvinar lectus consectetur pellentesque.

\paragraph*{S1 File.}
\label{S1_File}
{\bf Lorem ipsum.}  Maecenas convallis mauris sit amet sem ultrices gravida. Etiam eget sapien nibh. Sed ac ipsum eget enim egestas ullamcorper nec euismod ligula. Curabitur fringilla pulvinar lectus consectetur pellentesque.

\paragraph*{S1 Video.}
\label{S1_Video}
{\bf Lorem ipsum.}  Maecenas convallis mauris sit amet sem ultrices gravida. Etiam eget sapien nibh. Sed ac ipsum eget enim egestas ullamcorper nec euismod ligula. Curabitur fringilla pulvinar lectus consectetur pellentesque.

\paragraph*{S1 Appendix.}
\label{S1_Appendix}
{\bf Lorem ipsum.} Maecenas convallis mauris sit amet sem ultrices gravida. Etiam eget sapien nibh. Sed ac ipsum eget enim egestas ullamcorper nec euismod ligula. Curabitur fringilla pulvinar lectus consectetur pellentesque.

\paragraph*{S1 Table.}
\label{S1_Table}
{\bf Lorem ipsum.} Maecenas convallis mauris sit amet sem ultrices gravida. Etiam eget sapien nibh. Sed ac ipsum eget enim egestas ullamcorper nec euismod ligula. Curabitur fringilla pulvinar lectus consectetur pellentesque.

\section*{Acknowledgments}
Thanks to Alex Fornito, for being a big deal.\\
Should keep this!
\nolinenumbers

% Either type in your references using
% \begin{thebibliography}{}
% \bibitem{}
% Text
% \end{thebibliography}
%
% or
%
% Compile your BiBTeX database using our plos2015.bst
% style file and paste the contents of your .bbl file
% here. See http://journals.plos.org/plosone/s/latex for 
% step-by-step instructions.
% 

\bibliography{library}
% \begin{thebibliography}{10}

% \bibitem{bib1}
% Conant GC, Wolfe KH.
% \newblock {{T}urning a hobby into a job: how duplicated genes find new
%   functions}.
% \newblock Nat Rev Genet. 2008 Dec;9(12):938--950.
%
% \bibitem{bib2}
% Ohno S.
% \newblock Evolution by gene duplication.
% \newblock London: George Alien \& Unwin Ltd. Berlin, Heidelberg and New York:
%   Springer-Verlag.; 1970.
%
% \bibitem{bib3}
% Magwire MM, Bayer F, Webster CL, Cao C, Jiggins FM.
% \newblock {{S}uccessive increases in the resistance of {D}rosophila to viral
%   infection through a transposon insertion followed by a {D}uplication}.
% \newblock PLoS Genet. 2011 Oct;7(10):e1002337.

% \end{thebibliography}



\end{document}

